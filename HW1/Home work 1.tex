%% ================================================================================
%% This LaTeX file was created by AbiWord.                                         
%% AbiWord is a free, Open Source word processor.                                  
%% More information about AbiWord is available at http://www.abisource.com/        
%% ================================================================================

\documentclass[letterpaper,portrait,12pt]{article}
\usepackage[latin1]{inputenc}
\usepackage{calc}
\usepackage{setspace}
\usepackage{fixltx2e}
\usepackage{graphicx}
\usepackage{multicol}
\usepackage[normalem]{ulem}
%% Please revise the following command, if your babel
%% package does not support en-US
\usepackage[en]{babel}
\usepackage{color}
\usepackage{hyperref}
 
\begin{document}

\setlength{\oddsidemargin}{0.6250in-1in}

\begin{center}

\end{center}


\begin{center}

\end{center}


\begin{center}

\end{center}


\begin{center}

\end{center}


\begin{center}
{\huge CSE548/AMS542 Fall 2013 Analysis of Algorithms}
\end{center}


\begin{center}

\end{center}


\begin{center}
{\huge Hom}{\huge ework 1}
\end{center}


\begin{center}
{\huge Himanshu Shah}
\end{center}


\begin{center}
{\huge ID : 1093}{\huge 24380}
\end{center}


\begin{center}

\end{center}


\begin{center}

\end{center}


\begin{center}

\end{center}


\begin{center}

\end{center}


\begin{center}

\end{center}


\begin{center}

\end{center}


\begin{center}

\end{center}


\begin{center}

\end{center}


\begin{center}

\end{center}


\begin{center}

\end{center}


\begin{center}

\end{center}


\begin{center}

\end{center}


\begin{center}

\end{center}


\begin{center}

\end{center}


\begin{center}

\end{center}


\begin{center}

\end{center}


\begin{center}

\end{center}


\begin{center}
{\huge Hom}{\huge ework 1}
\end{center}


\begin{flushleft}
Problem 1:
\end{flushleft}


\begin{flushleft}

\end{flushleft}


\begin{flushleft}
\textbf{A) }\textbf{Ch}\textbf{-}\textbf{2 }\textbf{Probl}\textbf{em 4}\textbf{ }\textbf{[}\textbf{KT}\textbf{]}\textbf{:}
\end{flushleft}


\begin{flushleft}
\textbf{	}\textbf{Take the following list of functions and arrange them in ascending order}\textbf{ }\textbf{of growth rate. That is, if function g(n) immediately follows function f(n)}\textbf{ i}\textbf{n your list, then it should be the case that f(n) is O(g(n)).}
\end{flushleft}


\begin{flushleft}

\end{flushleft}


\begin{flushleft}
$(a)g_1(x)=2^\sqrt{logn} 
$
\end{flushleft}


\begin{flushleft}
$(b)g_2(x)=2^n 
$
\end{flushleft}


\begin{flushleft}
$(c)g_4(x)=n^{4/3}



$
\end{flushleft}


\begin{flushleft}
$(d)g_3(x)=n(logn)^3$
\end{flushleft}


\begin{flushleft}
$(e)g_5(x)=n^{logn}
$
\end{flushleft}


\begin{flushleft}
$(f)g_6(x)=2^{2^{n}}$
\end{flushleft}


\begin{flushleft}
$(g)g_7(x)=2^{n^{2}}$
\end{flushleft}


\begin{flushleft}

\end{flushleft}


\begin{flushleft}
Solution:
\end{flushleft}


\begin{flushleft}
It can be easily noted that $g_7(x)=O(g_6(x))$ and $g_2(x)=O(g_7(x))$. Also on taking the value of n=4, 16 and 64 the following order of growth can be seen:
\end{flushleft}


\begin{flushleft}
$g_1(x)=O(g_4(x))$
\end{flushleft}


\begin{flushleft}
$g_4(x)=O(g_3(x))
$
\end{flushleft}


\begin{flushleft}
$g_3(x)=O(g_5(x))
$
\end{flushleft}


\begin{flushleft}
$g_5(x)=O(g_2(x))
$
\end{flushleft}


\begin{flushleft}
Hence the growth of the functions in increasing order is :
\end{flushleft}


\begin{flushleft}

\end{flushleft}


\begin{flushleft}
$g_1(x), g_4(x), g_3(x), g_5(x), g_2(x), g_7(x) and g_6(x). With g_1(x)$  being the lowest and $g_6(x) $ being the largest
\end{flushleft}


\begin{flushleft}

\end{flushleft}


\begin{flushleft}

\end{flushleft}


\begin{flushleft}
\textbf{B}\textbf{) Ch-2 Problem }\textbf{5}\textbf{ [KT]: }
\end{flushleft}


\begin{flushleft}
\textbf{	}\textbf{Assume you have functions f and g such that f(n) is O(g(n)). For each of}\textbf{ }\textbf{the following statements, decide whether you think it is true or false and}\textbf{ }\textbf{give a proof or counterexample.}
\end{flushleft}


\begin{flushleft}
\textbf{(i) }$log_2 f(n)$\textbf{ is }$O(log_2g(n))$
\end{flushleft}


\begin{flushleft}
\textbf{S}\textbf{olution}\textbf{:}
\end{flushleft}


\begin{flushleft}
\textbf{True}
\end{flushleft}


\begin{flushleft}
\textbf{-}\textbf{ }This is evident from the fact that Log is an increasing function. Hence if $x$ $>$ $y$ then $log_2(x)$ will be greater than $log_2(y)$.
\end{flushleft}


\begin{flushleft}

\end{flushleft}


\begin{flushleft}
f(n) is O(g(n)) so, let f(n) = $x^2$ and g(n) = $x^3$.
\end{flushleft}


\begin{flushleft}

\end{flushleft}


\begin{flushleft}
Now $x^2 \le c  x^3$ 
\end{flushleft}


\begin{flushleft}
Applying $log_2$ on both sides
\end{flushleft}


\begin{flushleft}
$log_2(x^2) \le clog_2(x^3)$
\end{flushleft}


\begin{flushleft}
Hence, proved.
\end{flushleft}


\begin{flushleft}
(ii) $2^{f(n)}$ is $O(2^{g(n)})$
\end{flushleft}


\begin{flushleft}
-\textbf{True}
\end{flushleft}


\begin{flushleft}
The proof for this is pretty obvious, Considering the fact that exponent function is an increasing function.
\end{flushleft}


\begin{flushleft}
f(n) is O(g(n)) so, let f(n) = $x^2$ and g(n) = $x^3$.
\end{flushleft}


\begin{flushleft}

\end{flushleft}


\begin{flushleft}
So, $x^2 \le c  x^3$ 
\end{flushleft}


\begin{flushleft}

\end{flushleft}


\begin{flushleft}
Consider LHS = $2^{x^2}$ and RHS = $2^{x^{3}} $ multiplied by some constant m.
\end{flushleft}


\begin{flushleft}

\end{flushleft}


\begin{flushleft}
as $x^2 \le c  x^3$
\end{flushleft}


\begin{flushleft}
$2^{x^{2}} \le  m2^{x^{3}} $
\end{flushleft}


\begin{flushleft}
Hence  $2^{f(n)}$is $O(2^{g(n)})$. 
\end{flushleft}


\begin{flushleft}
Thus, proved.
\end{flushleft}


\begin{flushleft}
(iii) ${f(n)}^2$ is $O({g(n)}^2)$
\end{flushleft}


\begin{flushleft}
- \textbf{True}
\end{flushleft}


\begin{flushleft}
- Here we will prove using the property of Big-Oh function.
\end{flushleft}


\begin{flushleft}
According to the rule of Big-Oh multiplication by a function,
\end{flushleft}


\begin{flushleft}
$O(f(n)) O(g(n)) = O(f(n)g(n))$
\end{flushleft}


\begin{flushleft}

\end{flushleft}


\begin{flushleft}
So, as $f(n) is O(g(n))
$
\end{flushleft}


\begin{flushleft}
if we multiple $f(n) 
$ with $f(n) 
$ we get a 
\end{flushleft}


\begin{flushleft}
$f(n) ^2 = O(g(n)g(n))
$
\end{flushleft}


\begin{flushleft}
$f(n) ^2 = O(g(n)^2)
$
\end{flushleft}


\begin{flushleft}

\end{flushleft}


\begin{flushleft}
\textbf{C}\textbf{) Ch-2 Problem }\textbf{8}\textbf{ [KT]:}
\end{flushleft}


\begin{flushleft}

\end{flushleft}


\begin{flushleft}
\textbf{Sol}\textbf{ution }\textbf{a}\textbf{)}
\end{flushleft}


\begin{flushleft}
This problem can be solved using the following strategy:
\end{flushleft}


\begin{flushleft}
Try dropping the jar at every even number of rugs starting from level L = 2 and then increasing to $L_new=L_current+2$, until, the jar breaks.
\end{flushleft}


\begin{flushleft}
Go one level lower $L_new$=$L_current-1$, if the jar breaks, its highest safe rug level is $L_current-1.$
\end{flushleft}


\begin{flushleft}
		Else it is $L_current$.
\end{flushleft}


\begin{flushleft}

\end{flushleft}


\begin{flushleft}
Example suppose a jar has the safe level of 6. Then, as per the strategy, one should try dropping it at level 2 , 4 ,6 and then 8. The jar will break on 8. So try level 7, and it will break again and hence proving that 6 is its highest safe rug level.
\end{flushleft}


\begin{flushleft}

\end{flushleft}


\begin{flushleft}
\textbf{Solution }\textbf{b}\textbf{)}
\end{flushleft}


\begin{flushleft}
	Solution for question b is similar to the first one. Infact the solution a is a special case of solution b where number of jars =2.
\end{flushleft}


\begin{flushleft}
For this problem, I am considering k number of jars that can be used at maximum and $f_k(x)$ is the number of times an attempt is made. Here, n is the total number of rugs in the ladder.
\end{flushleft}


\begin{flushleft}

\end{flushleft}


\begin{flushleft}

\end{flushleft}


\begin{flushleft}
The strategy is as follows.
\end{flushleft}


\begin{flushleft}

\end{flushleft}


\begin{flushleft}
Step-1 Based on k, test the strength of the jar at m * k level starting m from 1 and incrementing it by one until the jar breaks.
\end{flushleft}


\begin{flushleft}

\end{flushleft}


\begin{flushleft}
Step-2 Test the strength at one level less than the previous height and see if it breaks. continue until it stops breaking.
\end{flushleft}


\begin{flushleft}

\end{flushleft}


\begin{flushleft}
This level is the highest safest level for the jar.
\end{flushleft}


\begin{flushleft}

\end{flushleft}


\begin{flushleft}
$>$ This algorithm will grow asymptotically lower than as the value of k will increase.
\end{flushleft}


\begin{flushleft}

\end{flushleft}


\begin{flushleft}
For k=3, n=10 and the highest safest level =7, this algorithm will work as:
\end{flushleft}


\begin{flushleft}
 $>$ First attempt : 3
\end{flushleft}


\begin{flushleft}
 $>$ Second attempt : 6
\end{flushleft}


\begin{flushleft}
 $>$ Third attempt : 9 - breaks
\end{flushleft}


\begin{flushleft}
	Fourth attempt : 8 - breaks
\end{flushleft}


\begin{flushleft}
	Fifth attempt : 7 - does not break - SOLUTION Found
\end{flushleft}


\begin{flushleft}

\end{flushleft}


\begin{flushleft}
For k=4, n=10 and the highest safest level =7, this algorithm will work as:
\end{flushleft}


\begin{flushleft}
 $>$ First attempt : 4
\end{flushleft}


\begin{flushleft}
 $>$ Second attempt : 8 - breaks
\end{flushleft}


\begin{flushleft}
	Third attempt : 7 - does not break - SOLUTION Found
\end{flushleft}


\begin{flushleft}

\end{flushleft}


\begin{flushleft}
Problem 2. Prove or disprove (i.e., give counter examples) for the following claims. f (n), g(n) are non- negative functions.
\end{flushleft}


\begin{flushleft}
$(a) max(f (n), g(n)) = \Theta(f (n) + g(n)).$
\end{flushleft}


\begin{flushleft}

\end{flushleft}


\begin{flushleft}

\end{flushleft}


\begin{flushleft}

\end{flushleft}


\begin{flushleft}

\end{flushleft}


\begin{flushleft}

\end{flushleft}


\begin{flushleft}
$(b) o(f (n)) \cap \omega(f (n)) = \emptyset$
\end{flushleft}


\begin{flushleft}

\end{flushleft}


\begin{flushleft}

\end{flushleft}


\begin{flushleft}

\end{flushleft}


\begin{flushleft}

\end{flushleft}


\begin{flushleft}

\end{flushleft}


\begin{flushleft}

\end{flushleft}


\begin{flushleft}

\end{flushleft}


\begin{flushleft}

\end{flushleft}


\begin{flushleft}

\end{flushleft}


\begin{flushleft}

\end{flushleft}


\begin{flushleft}

\end{flushleft}


\begin{flushleft}

\end{flushleft}


\begin{flushleft}

\end{flushleft}


\begin{flushleft}

\end{flushleft}


\begin{flushleft}

\end{flushleft}


\begin{flushleft}
$(c) (n + a)b = \Theta(nb ), a, b$ are positive integers
\end{flushleft}


\begin{flushleft}

\end{flushleft}


\begin{flushleft}

\end{flushleft}


\begin{flushleft}
$(d) f (n) = O(f (n)^2 )$
\end{flushleft}


\begin{flushleft}

\end{flushleft}


\begin{flushleft}

\end{flushleft}


\begin{flushleft}

\end{flushleft}


\begin{flushleft}

\end{flushleft}


\begin{flushleft}

\end{flushleft}


\begin{flushleft}

\end{flushleft}


\begin{flushleft}
$(e) f (n) = O(g(n))$ implies that $2f (n) = O(2g(n) )$
\end{flushleft}


\begin{flushleft}

\end{flushleft}


\begin{flushleft}

\end{flushleft}


\begin{center}

\end{center}


\begin{center}

\end{center}


\begin{flushleft}
Problem 3:
\end{flushleft}


\begin{flushleft}

\end{flushleft}


\begin{flushleft}

\end{flushleft}


\begin{flushleft}

\end{flushleft}


\begin{flushleft}

\end{flushleft}


\begin{flushleft}

\end{flushleft}


\begin{flushleft}

\end{flushleft}


\begin{flushleft}

\end{flushleft}


\begin{flushleft}

\end{flushleft}


\begin{flushleft}

\end{flushleft}


\begin{flushleft}

\end{flushleft}


\begin{flushleft}

\end{flushleft}


\begin{flushleft}

\end{flushleft}


\begin{flushleft}

\end{flushleft}


\begin{flushleft}

\end{flushleft}


\begin{flushleft}

\end{flushleft}


\begin{flushleft}

\end{flushleft}


\begin{flushleft}

\end{flushleft}


\begin{flushleft}

\end{flushleft}


\begin{flushleft}

\end{flushleft}


\begin{flushleft}

\end{flushleft}


\begin{flushleft}

\end{flushleft}


\begin{flushleft}

\end{flushleft}


\begin{flushleft}

\end{flushleft}


\begin{flushleft}

\end{flushleft}


\begin{flushleft}

\end{flushleft}


\begin{flushleft}

\end{flushleft}


\begin{flushleft}

\end{flushleft}


\begin{flushleft}

\end{flushleft}


\begin{flushleft}
	
\end{flushleft}


\begin{flushleft}
\textbf{Problem 4:}
\end{flushleft}


\begin{flushleft}
\textbf{	}\textbf{Given n unsorted numbers, compute the maximum and minimum using ⌊3n/2⌋$-$2 comparisons.}
\end{flushleft}


\begin{flushleft}

\end{flushleft}


\begin{flushleft}
Solution:
\end{flushleft}


\begin{flushleft}

\end{flushleft}


\begin{flushleft}
The algorithm to compute the maximum and minimum on n elements in ⌊3n/2⌋$-$2 comparisons can be done as follows :
\end{flushleft}


\begin{flushleft}

\end{flushleft}


\begin{flushleft}
Step -1 :
\end{flushleft}


\begin{flushleft}
Compare each of the even numbered element with the consecutive element in the list and determine the one that is greater. Add the elements that are larger in a set and name it SET A and the ones that are smaller in another SET B.
\end{flushleft}


\begin{flushleft}

\end{flushleft}


\begin{flushleft}
Step -2 :
\end{flushleft}


\begin{flushleft}
For each element in SET A find the maximum by performing a linear search.
\end{flushleft}


\begin{flushleft}

\end{flushleft}


\begin{flushleft}
Step - 3:
\end{flushleft}


\begin{flushleft}
For each element in SET B find the minimum by performing a linear search.
\end{flushleft}


\begin{flushleft}

\end{flushleft}


\begin{flushleft}
At the end of the algorithm, we will be able to determine the maximum and minimum from a given set of n numbers.
\end{flushleft}


\begin{flushleft}

\end{flushleft}


\begin{flushleft}
Calculating the number of comparisons made in each step of the algorithm:
\end{flushleft}


\begin{flushleft}

\end{flushleft}


\begin{flushleft}
Step -1 : For a given set of n numbers in order to generate two sets for larger and smaller numbers, one needs to make ⌊n/2⌋ Comparisons.
\end{flushleft}


\begin{flushleft}

\end{flushleft}


\begin{flushleft}
Step -2 :  In order to find the maximum from SET A using linear searching technique, one needs to perform, ⌊n/2⌋ -1 comparisons. This is because, the number of elements in the set is ⌊n/2⌋. So, the number of comparisons required is one less than the total number of elements in the set.
\end{flushleft}


\begin{flushleft}

\end{flushleft}


\begin{flushleft}
Step -3 :  In order to find the minimum from SET B using linear searching technique, one needs to perform, ⌊n/2⌋ -1 comparisons. This is because, the number of elements in the set is ⌊n/2⌋. So, the number of comparisons required is one less than the total number of elements in the set.
\end{flushleft}


\begin{flushleft}

\end{flushleft}


\begin{flushleft}
Hence, total number of comparisions is  ⌊n/2⌋ +  ⌊n/2⌋ -1 +  ⌊n/2⌋ -1 
\end{flushleft}


\begin{flushleft}
= 3 ⌊n/2⌋ -2
\end{flushleft}


\begin{flushleft}
= ⌊3n/2⌋-2
\end{flushleft}


\begin{flushleft}
 
\end{flushleft}


\begin{flushleft}
Hence, proved.
\end{flushleft}


\begin{flushleft}

\end{flushleft}


\begin{flushleft}

\end{flushleft}


\begin{flushleft}
 
\end{flushleft}


\begin{flushleft}

\end{flushleft}


\begin{flushleft}

\end{flushleft}


\begin{flushleft}

\end{flushleft}


\begin{flushleft}

\end{flushleft}


\begin{flushleft}
\textbf{Problem }\textbf{5}\textbf{:}
\end{flushleft}


\begin{flushleft}
\textbf{Page 107 Problem \#4 :}
\end{flushleft}


\begin{flushleft}

\end{flushleft}


\begin{flushleft}

\end{flushleft}


\begin{flushleft}

\end{flushleft}


\begin{flushleft}

\end{flushleft}


\begin{flushleft}

\end{flushleft}


\begin{flushleft}

\end{flushleft}


\begin{flushleft}

\end{flushleft}


\begin{flushleft}

\end{flushleft}


\begin{flushleft}

\end{flushleft}


\begin{flushleft}

\end{flushleft}


\begin{flushleft}

\end{flushleft}


\begin{flushleft}

\end{flushleft}


\begin{flushleft}

\end{flushleft}


\begin{flushleft}

\end{flushleft}


\begin{flushleft}

\end{flushleft}


\begin{flushleft}

\end{flushleft}


\begin{flushleft}

\end{flushleft}


\begin{flushleft}

\end{flushleft}


\begin{flushleft}

\end{flushleft}


\begin{flushleft}

\end{flushleft}


\begin{flushleft}

\end{flushleft}


\begin{flushleft}

\end{flushleft}


\begin{flushleft}

\end{flushleft}


\begin{flushleft}

\end{flushleft}


\begin{flushleft}

\end{flushleft}


\begin{flushleft}

\end{flushleft}


\begin{flushleft}

\end{flushleft}


\begin{flushleft}

\end{flushleft}


\begin{flushleft}

\end{flushleft}


\begin{flushleft}
\textbf{Page 107 Problem \#}\textbf{6}\textbf{ :}
\end{flushleft}


\begin{flushleft}

\end{flushleft}


\begin{flushleft}

\end{flushleft}


\begin{flushleft}

\end{flushleft}


\begin{flushleft}

\end{flushleft}


\begin{flushleft}

\end{flushleft}


\begin{flushleft}

\end{flushleft}


\begin{flushleft}

\end{flushleft}


\begin{flushleft}

\end{flushleft}


\begin{flushleft}

\end{flushleft}


\begin{flushleft}

\end{flushleft}


\begin{flushleft}

\end{flushleft}


\begin{flushleft}

\end{flushleft}


\begin{flushleft}

\end{flushleft}


\begin{flushleft}

\end{flushleft}


\begin{flushleft}

\end{flushleft}


\begin{flushleft}

\end{flushleft}


\begin{flushleft}

\end{flushleft}


\begin{flushleft}

\end{flushleft}


\begin{flushleft}

\end{flushleft}


\begin{flushleft}

\end{flushleft}


\begin{flushleft}

\end{flushleft}


\begin{flushleft}

\end{flushleft}


\begin{flushleft}
\textbf{Page 107 Problem \#}\textbf{10}\textbf{ :}
\end{flushleft}


\begin{flushleft}

\end{flushleft}


\begin{flushleft}

\end{flushleft}


\begin{flushleft}

\end{flushleft}


\begin{center}

\end{center}


\begin{center}

\end{center}


\begin{center}

\end{center}




8
\end{document}
