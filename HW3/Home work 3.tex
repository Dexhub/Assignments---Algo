\documentclass[12pt]{article}
\usepackage{url,graphicx,tabularx,array,geometry}
\setlength{\parskip}{1ex} %--skip lines between paragraphs
\setlength{\parindent}{0pt} %--don't indent paragraphs

%-- Commands for header
\renewcommand{\title}[1]{\textbf{#1}\\}
\renewcommand{\line}{\begin{tabularx}{\textwidth}{X>{\raggedleft}X}\hline\\\end{tabularx}\\[-0.5cm]}
\newcommand{\leftright}[2]{\begin{tabularx}{\textwidth}{X>{\raggedleft}X}#1%
& #2\\\end{tabularx}\\[-0.5cm]}

%\linespread{2} %-- Uncomment for Double Space
\begin{document}

\title{Home work 3}
\line
\leftright{\today}{Himanshu Shah} %-- left and right positions in the header

\section{Problem 2 - pg 189}
\textbf{(a) True}.

When we square the values of all edges, they increase proportionately. Hence, the values that are smaller will remain smaller and those that are bigger will still remain bigger comparitively. In an MST only the edges with minimum weights are present. Hence,the MST will still have the same set of edges and will remain the same.

\textbf{(b)True}

Here also the same argument applies. As we are squaring the value of each of the edges, in the graph, the order of the edges in terms of their weights still remains unaffected and hence the path which that had the minimum cost between points s-t will still be the same.  

\section{Problem 20 - pg 199}
\textbf{(i) True}
By definition, the minimum spanning tree is a spanning tree with weight less than or ewq


What I do more often is simply put the ``section'' titles in a
$\backslash$textbf\{\} tag (as shown in this section).  This makes them stand
out, but doesn't take up a lot of extra white space on the page.

\textbf{All the benefits of \LaTeX}

Using \LaTeX for homeworks allows you to do lots of nice things like:
\begin{itemize}
\item Have your work neatly typeset
\item Impress your professors who might recognize the fonts
\item Include equations easily
\item Put math $\in$ the text: $\textrm{money} = \sqrt{\textrm{all evil}}$
\item Not ever have to mess with MS Word
\end{itemize}

Being able to include equations is especially nice.  Say you were doing a
homework the Pythagorean theorem:
\begin{equation}
d = \sqrt{a^2+b^2}.
\end{equation}
It is simple to include it right in the text!  In fact, when doing math
homeworks I often find it simpler to code derivations in \LaTeX.  If I make a
mistake somewhere, I can quickly change all down-line equations with a
find-replace procedure instead of having to re-write or cross out large blocks
of text.

\end{document}
